\documentclass[11pt]{article}

\usepackage[margin=1in]{geometry}
\usepackage{graphicx}
\usepackage{booktabs}
\usepackage{amsmath}
\usepackage{siunitx}
\usepackage{hyperref}
\usepackage{microtype}

\title{Benchmarking Differentially Private PrivBayes Variants on a Medical Classification Dataset}
\author{(Draft) privbayes-benchmark}
\date{\today}

\begin{document}
\maketitle

\begin{abstract}
We benchmark three implementations of a PrivBayes-style differentially private (DP) tabular synthesizer---\textbf{Enhanced}, \textbf{SynthCity}, and \textbf{DPMM}---on a medical classification dataset derived from the UCI Breast Cancer Wisconsin diagnostic dataset.
We evaluate utility (set-based similarity, mutual information preservation, distributional distances, correlation preservation, and downstream ML performance), privacy (QI linkage rate; exact row match rate), and performance (runtime and memory).
Across five random seeds and privacy budgets \(\varepsilon \in \{0.1, 0.5, 1.0, 3.0, 5.0\}\), we observe substantial trade-offs: SynthCity achieves markedly higher similarity-based utility but at very high memory cost, while Enhanced is orders-of-magnitude faster with minimal memory footprint but substantially lower similarity-based utility; privacy leakage via QI linkage is similar across mechanisms in this dataset.
\end{abstract}

\section{Introduction}
Synthetic data generation under DP aims to enable data sharing and analysis while bounding privacy risk.
PrivBayes-style models synthesize tabular data by learning a Bayesian network structure (under DP) and sampling from a noisy factorization.
Despite conceptual similarity, practical implementations can differ significantly in their preprocessing, discretization, optimization, and engineering choices, leading to large differences in utility, privacy, and performance.
This paper reports a reproducible benchmark using the \texttt{privbayes-benchmark} pipeline and summarizes conclusions for a medical classification task.

\section{Methods}
\subsection{Mechanisms}
We evaluate three mechanisms exposed by the benchmark:
\begin{itemize}
  \item \textbf{Enhanced}: a lightweight optimized PrivBayes implementation.
  \item \textbf{SynthCity}: PrivBayes via the \texttt{synthcity} library.
  \item \textbf{DPMM}: a PrivBayes variant provided by the \texttt{dpmm} package.
\end{itemize}

\subsection{Privacy budgets and randomness}
Each mechanism is run at privacy budgets \(\varepsilon \in \{0.1, 0.5, 1.0, 3.0, 5.0\}\), using \textbf{5} random seeds per \((\text{mechanism}, \varepsilon)\) pair.
All plots show the \textbf{mean} across seeds.
Uncertainty bands are plotted as a normal-approximation 95\% confidence interval:
\[
\bar{x} \pm 1.96 \cdot \mathrm{SE}, \quad \mathrm{SE}=\frac{s}{\sqrt{n}}, \; n=5.
\]
(\emph{Note}: with \(n=5\), a \(t\)-interval is also reasonable; we use the same convention as the plotting script.)

\subsection{Dataset}
We use a CSV export of the Breast Cancer Wisconsin diagnostic dataset with a binary target label (\texttt{target}).
The benchmark treats the dataset as tabular and computes both statistical similarity metrics and downstream ML performance.

\subsection{Metrics}
\paragraph{Utility.}
The 9-panel figure reports: Weighted Jaccard similarity, privacy-budget efficiency (utility per \(\varepsilon\)), mutual information preservation, distributional distances (TVD for 1D/2D/3D marginals; EMD), correlation preservation (Pearson/Spearman matrix agreement), and downstream ML AUC (Syn\(\rightarrow\)Real) for Logistic Regression (LR) and Random Forest (RF).

\paragraph{Privacy.}
We report QI linkage rate (lower is better). Exact row match rate (ERMR) is also computed but is constant at \(\textbf{0}\) across all runs here.

\paragraph{Performance.}
We report total runtime, plus any available fit/sample time breakdown, and peak memory.

\section{Experimental Results}
\subsection{Figures}
Figure~\ref{fig:utility_privacy} summarizes utility, privacy, and selected performance measures across \(\varepsilon\).
Figure~\ref{fig:performance} breaks out performance into four panels.

\begin{figure}[t]
  \centering
  \includegraphics[width=\textwidth]{medical_breast_cancer_trend_eps_seeds5/trend_eps_seeds5_ci95_v5_pubtitle.pdf}
  \caption{Utility, privacy, and performance summary across privacy budgets \(\varepsilon \in \{0.1,0.5,1.0,3.0,5.0\}\) (mean across 5 seeds; error bars are 95\% CI). Panels (a)--(i) are discussed in Section~\ref{sec:panelwise}.}
  \label{fig:utility_privacy}
\end{figure}

\begin{figure}[t]
  \centering
  \includegraphics[width=\textwidth]{medical_breast_cancer_trend_eps_seeds5/trend_eps_seeds5_ci95_v5_pubtitle_performance.pdf}
  \caption{Performance breakdown (mean across 5 seeds; 95\% CI): total time, fit time, sample time, and peak memory vs \(\varepsilon\). Panels (a)--(d).}
  \label{fig:performance}
\end{figure}

\subsection{Panel-by-panel analysis and conclusions}
\label{sec:panelwise}

Below we interpret each panel in Figure~\ref{fig:utility_privacy}, then discuss Figure~\ref{fig:performance}.
When we cite representative numbers, they are computed from the same JSON used to create the figures (5 seeds per setting).

\subsubsection*{Panel (a): Utility vs privacy budget (Weighted Jaccard; higher is better)}
SynthCity achieves substantially higher Weighted Jaccard than the other mechanisms across all \(\varepsilon\) values.
For example, at \(\varepsilon=1.0\), SynthCity is approximately \(0.472 \pm 0.001\) (95\% CI), while Enhanced is \(0.0181 \pm 0.0004\), and DPMM is \(0.0135 \pm 0.0019\).
\textbf{Conclusion:} on this dataset and metric, SynthCity is the clear utility leader, whereas Enhanced and DPMM are far lower and close to each other.

\subsubsection*{Panel (b): Privacy-budget efficiency (utility per \(\varepsilon\); higher is better)}
Because SynthCity dominates absolute utility, it also dominates utility-per-\(\varepsilon\) for most \(\varepsilon\) values.
\textbf{Conclusion:} increasing \(\varepsilon\) does not yield comparable improvements for Enhanced/DPMM under this particular similarity metric; their efficiency curves are relatively flat and low.

\subsubsection*{Panel (c): QI linkage rate (lower is better; ERMR note)}
ERMR is constant at \(0\) for all mechanisms and all budgets, indicating no exact row matches between real and synthetic rows in these runs.
QI linkage rates are broadly similar across mechanisms and do not show a strong monotonic decrease with \(\varepsilon\) in this dataset.
Representative values at \(\varepsilon=1.0\) are \(\approx 0.376\)--\(0.385\) across mechanisms (with overlapping 95\% CIs).
\textbf{Conclusion:} within this evaluation, privacy risk as captured by QI linkage is comparable across implementations; utility differences are not accompanied by large linkage differences.

\subsubsection*{Panel (d): Total time vs \(\varepsilon\) (log scale; lower is better)}
Enhanced is orders of magnitude faster than SynthCity and DPMM (sub-second total time vs hundreds to thousands of seconds).
DPMM and SynthCity show large run-to-run variability in total time, especially at small \(\varepsilon\) (e.g., DPMM at \(\varepsilon=0.1\) has a very wide CI due to heavy-tailed runtimes).
\textbf{Conclusion:} Enhanced is the performance leader by runtime; SynthCity and DPMM incur substantial compute cost in this benchmark.

\subsubsection*{Panel (e): Utility--privacy tradeoff (Weighted Jaccard vs QI linkage; \(\varepsilon\) trajectory)}
This panel plots \textbf{utility} (y-axis) against \textbf{QI linkage} (x-axis), connecting points in increasing \(\varepsilon\) order for each mechanism.
Because QI linkage is similar across mechanisms while utility differs sharply, SynthCity occupies a higher-utility region with comparable linkage.
\textbf{Conclusion:} for these settings, the mechanisms differ far more in utility than in linkage, and SynthCity offers a better observed utility--linkage operating point (at substantial system cost; see Figure~\ref{fig:performance}).

\subsubsection*{Panel (f): Mutual information preservation (higher is better)}
MI preservation trends capture whether pairwise dependence structure is retained.
If SynthCity maintains high utility, it is typically consistent with higher dependence preservation, though MI curves may be less separated than Weighted Jaccard depending on discretization and binning effects.
\textbf{Conclusion:} MI preservation complements similarity metrics and helps diagnose whether utility gains reflect structural fidelity rather than chance overlap.

\subsubsection*{Panel (g): TVD/EMD (log scale; lower is better)}
TVD (1D/2D/3D) summarizes distributional mismatch at increasing marginal dimensionality; 3D TVD is typically the hardest.
A mechanism can achieve high similarity on one metric while still exhibiting higher-order distributional drift, which this panel reveals.
\textbf{Conclusion:} distributional distances provide a stricter view of fidelity than Jaccard-like similarity and should be interpreted jointly with Panels 1 and 6.

\subsubsection*{Panel (h): Correlation preservation (higher is better)}
This panel evaluates how well correlation structure is preserved (Pearson and Spearman agreement between real and synthetic correlation matrices).
\textbf{Conclusion:} correlation preservation is important for downstream modeling and feature interactions; a mechanism with high utility but poor correlation preservation may still underperform on predictive tasks.

\subsubsection*{Panel (i): Downstream ML (Syn\(\rightarrow\)Real AUC; higher is better)}
Downstream AUC measures whether synthetic training data supports predictive models that generalize to the real test distribution.
On this benchmark, downstream AUCs can have wide uncertainty (reflecting sensitivity to seed, preprocessing, and the synthetic label distribution), especially for DPMM and Enhanced.
For example, at \(\varepsilon=1.0\), SynthCity achieves LR AUC \(\approx 0.711 \pm 0.057\) and RF AUC \(\approx 0.654 \pm 0.062\), while other mechanisms have lower means and/or wider intervals.
\textbf{Conclusion:} SynthCity tends to yield stronger downstream performance at moderate \(\varepsilon\), but uncertainty can be large; increasing the number of seeds would tighten intervals and support stronger claims.

\subsection{Performance figure interpretation (Figure~\ref{fig:performance})}
\paragraph{Total time.}
Enhanced is consistently sub-second, while SynthCity and DPMM are on the order of minutes to hours across settings.

\paragraph{Fit vs sample time.}
If fit/sample breakdowns are present, they help attribute cost (e.g., expensive structure learning vs sampling).

\paragraph{Peak memory.}
SynthCity exhibits extremely high memory usage in this environment (tens of GB), while Enhanced is low (tens of MB) and DPMM is very low (single-digit MB).
\textbf{Conclusion:} mechanism choice is constrained not only by privacy/utility but by practical resource limits; SynthCity may be infeasible on modest hardware without careful configuration.

\section{Limitations and recommended next steps}
\begin{itemize}
  \item \textbf{Small-\(n\) uncertainty:} with 5 seeds, CIs can be noisy; repeating with 20+ seeds or using \(t\)-intervals would strengthen inferential claims.
  \item \textbf{Metric sensitivity:} similarity metrics depend on discretization and feature handling; reporting additional task-specific metrics can clarify what ``utility'' means for the intended use.
  \item \textbf{Hardware dependence:} runtime/memory are hardware- and library-version-dependent; pin and report environment details when comparing across machines.
\end{itemize}

\section{Overall conclusions}
On the Breast Cancer dataset under the benchmark’s metric suite:
\begin{itemize}
  \item \textbf{Utility:} SynthCity achieves the highest similarity-based utility and often strong downstream AUC at moderate \(\varepsilon\).
  \item \textbf{Privacy (observed):} QI linkage is similar across mechanisms here; ERMR is \(0\) throughout.
  \item \textbf{Performance:} Enhanced is dramatically faster and memory-efficient; SynthCity is memory-intensive; DPMM is slow but memory-light.
\end{itemize}
These results suggest that practical deployment depends on whether one prioritizes raw fidelity (SynthCity) versus lightweight generation (Enhanced), and whether available hardware can support high-memory synthesis.

\begin{thebibliography}{9}
\bibitem{privbayes}
Zhang, J., Cormode, G., Procopiuc, C.~M., Srivastava, D., and Xiao, X.
\newblock PrivBayes: Private Data Release via Bayesian Networks.
\newblock \emph{SIGMOD}, 2014.

\bibitem{synthcity}
Qian, Z. \emph{et al.}
\newblock SynthCity: A benchmark framework for tabular synthetic data generation.
\newblock (Software library; see \url{https://github.com/vanderschaarlab/synthcity}.)
\end{thebibliography}

\end{document}

